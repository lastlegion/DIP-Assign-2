%----------------------------------------------------------------------------------------
%	PACKAGES AND OTHER DOCUMENT CONFIGURATIONS
%----------------------------------------------------------------------------------------

\documentclass[paper=a4, fontsize=11pt]{scrartcl} % A4 paper and 11pt font size
\usepackage{graphicx} %For inserting pictures
\usepackage[T1]{fontenc} % Use 8-bit encoding that has 256 glyphs
\usepackage[english]{babel} % English language/hyphenation
\usepackage{amsmath,amsfonts,amsthm} % Math packages
\usepackage{float} %For H floats
%\allsectionsfont{\centering \normalfont\scshape} % Make all sections centered, the default font and small caps

\usepackage{fancyhdr} % Custom headers and footers
\pagestyle{fancyplain} % Makes all pages in the document conform to the custom headers and footers
\fancyhead{} % No page header - if you want one, create it in the same way as the footers below
\fancyfoot[L]{} % Empty left footer
\fancyfoot[C]{} % Empty center footer
\fancyfoot[R]{\thepage} % Page numbering for right footer
\renewcommand{\headrulewidth}{0pt} % Remove header underlines
\renewcommand{\footrulewidth}{0pt} % Remove footer underlines
\setlength{\headheight}{13.6pt} % Customize the height of the header

\numberwithin{equation}{section} % Number equations within sections (i.e. 1.1, 1.2, 2.1, 2.2 instead of 1, 2, 3, 4)
\numberwithin{figure}{section} % Number figures within sections (i.e. 1.1, 1.2, 2.1, 2.2 instead of 1, 2, 3, 4)
\numberwithin{table}{section} % Number tables within sections (i.e. 1.1, 1.2, 2.1, 2.2 instead of 1, 2, 3, 4)

\setlength\parindent{0pt} % Removes all indentation from paragraphs - comment this line for an assignment with lots of text
\DeclareGraphicsExtensions{.pdf,.png,.jpg}
\graphicspath{ {../images/} }
%----------------------------------------------------------------------------------------
%	TITLE SECTION
%----------------------------------------------------------------------------------------

\newcommand{\horrule}[1]{\rule{\linewidth}{#1}} % Create horizontal rule command with 1 argument of height

\title{	
\normalfont \normalsize 
\textsc{Dhirubhai Ambani Institute of Information and Communication Technology} \\ [25pt] % Your university, school and/or department name(s)
\horrule{0.5pt} \\[0.4cm] % Thin top horizontal rule
\huge Assignment 2 \\ % The assignment title
\horrule{2pt} \\[0.5cm] % Thick bottom horizontal rule
}

\author{Ganesh Iyer \\ 201311019 \\Developed using: Python(Using SimpleCV library)}

\date{\normalsize\today} % Today's date or a custom date

\begin{document}

\maketitle % Print the title

%----------------------------------------------------------------------------------------
%	PROBLEM 1
%----------------------------------------------------------------------------------------

\section{Bitplanes}

    \subsection{Bitplanes slicing}
    In this problem we implemented a function mybitplane() that extracts all 8 bit planes of any input grayscale image I. 
    \begin{figure}[h!]
        \centering
        \includegraphics[clip,height=3cm]{lena}
        \caption{Original Image}
    \end{figure}
    \begin{figure}[h!]
        \centering
        \includegraphics[clip,height=3cm]{BitPlane0}
        \includegraphics[clip,height=3cm]{BitPlane1}
        \includegraphics[clip,height=3cm]{BitPlane2}
        \includegraphics[clip,height=3cm]{BitPlane3}
        \includegraphics[clip,height=3cm]{BitPlane4}
        \includegraphics[clip,height=3cm]{BitPlane5}
        \includegraphics[clip,height=3cm]{BitPlane6}
        \includegraphics[clip,height=3cm]{BitPlane7}
        \caption{Bitplanes, Starting from Least significant bit(TopLeft) to Most significant bit in clockwise direction}
    \end{figure}
    As we can observe the most significant bits seem to have more information pertaining to the image
    \begin{figure}[h!]
        \centering
        \includegraphics[clip,height=3cm]{daiict}
        \caption{Watermark}
    \end{figure}
    \begin{figure}[h!]
        \centering
        \includegraphics[clip,height=3cm]{WatermarkLevel0}
        \includegraphics[clip,height=3cm]{WatermarkLevel1}
        \includegraphics[clip,height=3cm]{WatermarkLevel2}
        \includegraphics[clip,height=3cm]{WatermarkLevel3}
        \includegraphics[clip,height=3cm]{WatermarkLevel4}
        \includegraphics[clip,height=3cm]{WatermarkLevel5}
        \includegraphics[clip,height=3cm]{WatermarkLevel6}
        \includegraphics[clip,height=3cm]{WatermarkLevel7}
        \caption{WaterMark applied on different layers of the original image. Starting from level 0(top-left) to level 7 in a clock-wise direction}
    \end{figure}

    %------------------------------------------------

    \subsection{Watermarking}
    In this exercise we use the binary image \texttt{daiict.bmp} that was provided as a watermark and replace the \(i^{th}\) bit plane of the image \texttt{lena.jpg} and reconstruct the gray scale image \(J_i\) for \(1 \leq i \leq 8\).
    %------------------------------------------------

\section{Histogram equalization}
    In this exercise we wrote a function \texttt{myhisteq()} that applies histogram equalization on any input grayscale image. 
    We use the transfomration function \\\(T(r_k) = (L-1)\sum_{j=0}^{k}p_R({r_j})\)

     \begin{figure}[h!]
        \centering
        \includegraphics[clip,height=3cm]{histeq-my}
        \includegraphics[clip,height=3cm]{histeq-scv}
        \caption{Clockwise starting from top left; Original Image; Image output of our histogram equalization; Image output of built-in histogram equalization}
    \end{figure}
    \begin{figure}[h!]
        \centering
        \includegraphics[clip,height=5cm]{hist-orig}
        \includegraphics[clip,height=5cm]{hist-my}
        \includegraphics[clip,height=5cm]{hist-eq}
        \caption{Clockwise starting from top left; Original Histogram; Output of our histogram equalization; Output of built-in histogram equalization}
    \end{figure}

\section{Convolution}
We implemeneted a function \texttt{my2Dconv()} that takes an Image(\(I\)) and kernel(\(k\)) and output \(y=I*k\).         
We use the following kernel:
\(k = \begin{pmatrix}
            1 & 2 & 1 \\
            0 & 0 & 0 \\
            -1 & -2 & -1
            \end{pmatrix}
\)
    \begin{figure}[h!]
        \centering
        \includegraphics[clip,height=3cm]{convolvemy}
        \includegraphics[clip,height=3cm]{convolvescv}
        \caption{Outputs using kernel(k). Left: Using my2DConv function. Right: Using built-in function}
    \end{figure}

\section{Deriving popular convolution masks}
Given a \(3X3\) neighbourhood(\(F\)) of an image 
    \\\(F = \begin{bmatrix}
            f(-1,-1) && f(-1,0) && f(-1,1) \\
            f(0,-1) && f(0,0) && f(0,1) \\ 
            f(1,-1) && f(1,0) && f(1,1) 
            \end{bmatrix}\)
    \\
    The equation of the actual plane
    \\ \(g(x,y) = a + bx + cy. \)
    Now we have \(F = G + \eta \)
    \subsection{Finding \(\beta\)}
    For this we use the \texttt{lstsq} function built-in Numpy for finding least square solution to a linear matrix equation.
    This function solves the equation \(F=X\beta\) by computing \(\beta\) such that it minimizes the Euclidean 2-norm \(\|F-X\beta\|^2\). \\
    We get the following values of a,b and c for the first \(3X3\) neighbourhood.
    \\\( \beta = 
        \begin{bmatrix}
            161.0 && -0.5 && -0.5
        \end{bmatrix}
    \)

    \subsection{Finding convolution masks}
    We find the pseudo-inverse of X using the built-in function \texttt{pinv}. We get the following matrices for our convolution masks
    \\\(M1 = \begin{bmatrix}
            0.1 && 0.1 && 0.1 \\ 0.1 && 0.1 && 0.1 \\ 0.1 && 0.1 && 0.1

        \end{bmatrix}
      \\
      M2 = \begin{bmatrix}
           -0.2 && -0.2 && -0.2 \\ 0 && 0 && 0 \\ 0.2 && 0.2 && 0.2
          \end{bmatrix}
      \\
      M3 = \begin{bmatrix}
          -0.2 && 0 && 0.2 \\ -0.2 && 0 && 0.2 \\ -0.2 && 0 && 0.2
            \end{bmatrix}
     \)
    \subsection{Finding convolution masks using error weights}
    We multiply the noise weights with original masks to obtain new masks.
    \\
    \(
      M_{w1} = \begin{bmatrix}
            0.4 && 0.9 && 0.4\\
            0.4 && 0.9 && 0.4\\
            0.4 && 0.9 && 0.4
          \end{bmatrix}
      \\
      M_{w2} = \begin{bmatrix}
            -0.67 && -1.33 && -0.67\\
              0 && 0 && 0  \\
             0.67 && 1.33 && 0.67
          \end{bmatrix}
      \\
      M_{w3} = \begin{bmatrix}
            0 &&  0 && 0 \\ 
            0 && 0 &&  0 \\ 
            0 && 0 &&  0
          \end{bmatrix}
    \)
    \subsection{Results}

    \begin{figure}[h!]
        \centering
        \includegraphics[clip,height=3cm]{conv1}
        \includegraphics[clip,height=3cm]{conv2}
        \includegraphics[clip,height=3cm]{conv3}

        \includegraphics[clip,height=3cm]{convw1}
        \includegraphics[clip,height=3cm]{convw2}
        \includegraphics[clip,height=3cm]{convw3}

        \caption{Starting from top-left, results of applying the convolution masks dervied(\(M1,M2,M3,M_{w1}, M_{w2}, M_{w3}\)) in clockwise direction. }
    \end{figure}
        We observe that these masks work like the Sobel masks for edge detection. Results of \(M2, M3\) are similar to Sobel mask in X and Y orientation.
\end{document}
